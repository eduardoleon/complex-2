\begin{exercise}
Sean $a_0, \dots, a_n$ enteros positivos tales que $\gcd(a_0, \dots, a_n) = 1$. Sea $\C^{n+1}$ con coordenadas complejas $z_0, \dots, z_n$. Considere la acción de $\C^\star$ sobre $\C^{n+1} - \{ 0 \}$ vía reescalamientos ponderados
$$\lambda \cdot (z_0, \dots, z_n) = (\lambda^{a_0} z_0, \dots, \lambda^{a_n} z_n)$$
El \textit{espacio proyectivo ponderado} $\CP^n[a_0, \dots, a_n]$ es el espacio de órbitas de esta acción.
\begin{enumerate}[label=\alph*)]
    \item Pruebe que $\CP^n[a_0, \dots, a_n]$ es un espacio compacto y Hausdorff.
    
    \item Observe que $\CP^n[1, \dots, 1]$ es el espacio proyectivo complejo usual $\CP^n$. Por tanto, los espacios proyectivos ponderados determinan una familia de espacios que generalizan las variedades complejas $\CP^n$. Muestre que los espacios proyectivos ponderados $\CP^n[a_0, \dots, a_n]$ no son variedades complejas (manifolds) de manera genérica. De hecho, estos espacios son ejemplos de \textit{orbifolds} (o $V$-\textit{manifolds}).
    
    \item Si $a_0$ y $a_2$ no fuesen primos relativos, ¿podría encontrar alguna vecindad del punto $[1:0:1:0:\dots:0]$ que sea homeomorfa a algún abierto de $\C^n$?
    
    \item ¿Podría usted encontrar condiciones sobre los pesos $a_i$ que determinen si ciertas vecindades son homeomorfas a abiertos de $\C^n$?
\end{enumerate}
\end{exercise}
